\documentclass[a4paper]{article}
\usepackage{a4wide}
\setlength{\parskip}{0.7ex plus0.1ex minus0.1ex}
\setlength{\parindent}{0em}

\usepackage{/usr/lib/R/share/texmf/Sweave}
\begin{document}


%\VignetteIndexEntry{Writing CRAN Task Views}
%\VignetteDepends{ctv,XML}
%\VignetteKeywords{CRAN task views, repository, packages, installation}
%\VignettePackage{ctv}

\title{Writing CRAN Task Views}
\author{Achim Zeileis}
\date{}
\maketitle

CRAN task views should provide some guidance which part
of the growing number of packages on CRAN are useful for
a certain topic. They should give a brief overview of the
contained packages and allow automatic installation. Infrastructure
for both tasks is provided in \verb/.ctv/ files, an XML-based
specification of CRAN task views. The format is based on just
a handful of new tags and is mostly self-explanatory. A brief
artificial example is given here, more detailed explanations
are given below:

\begin{verbatim}
<CRANTaskView>

  <name>Econometrics</name>
  <topic>Computational Econometrics &amp; Finance</topic>
  <maintainer>Achim Zeileis</maintainer>
  
  <info>
    Emphasize something <b>important</b> with standard HTML tags,
    and reference the package <pkg>foo</pkg> and also the view
    <view>MachineLearning</view>.
  </info>

  <packagelist>
    <pkg>foo</pkg>
    <pkg priority="core">bar</pkg>
  </packagelist>

  <links>
    <a href="http://path/to/homepage/">The Title of a Relevant Homepage</a>
    <view>MachineLearning</view>
  </links>
  
</CRANTaskView>
\end{verbatim}

The whole document is framed by an opening and closing \verb/<CRANTaskView>/
tag. Within these, there are six sections: \verb/<name>/, \verb/<topic>/,
\verb/<maintainer>/, \verb/<info>/, \verb/<packagelist>/ and \verb/<links>/.

\verb/<name>/ gives the name of the task view. This is used as the identifier
for installing the view and as the name for the XML file, e.g., \verb/Econometrics.ctv/,
and the auto-generated HTML file, e.g., \verb/Econometrics.html/. Hence, it should be
not too long and contain no special characters like spaces.

\verb/<topic>/ is a plain text specification of the topic of the task view.
(Note that in XML files there are three special characters \verb/</, \verb/>/
and \verb/&/ that need to be escaped by their HTML counterparts.)

\verb/<maintainer>/ gives the name of the maintainer in plain text.

\verb/<info>/ should be a short description of the packages that explains
which packages are useful for which tasks. It can also contain any further
informations and should be written in plain HTML code, i.e., any HTML markup
can be used. There are two additional tags available: \verb/<pkg>/ and 
\verb/<view>/ which should be used as markup for packages and views, respectively.
During generation of the corresponding HTML file, they will be replaced by
hyperlinks to the packages/views.

\verb/<packagelist>/ contains the list of packages associated with the view.
It should contain (at least) all packages mentioned in the \verb/<info>/ section
and allows for a distinction between \verb/"core"/ and \verb/"normal"/ (default)
priority packages. The distinction is only important for the installation of
certain views because the user can specify whether all packages (default) or only
the most important core packages should be installed (with all their dependencies).

/verb/<links>/ is a means of specifying further places where information about
the discussed topic and related issues is discussed, e.g., manuals/books/tutorials
on how to do Econometrics, say, with \textsf{R}.

For a template of a \verb/.ctv/ file, look at the included \verb/Econometrics.ctv/.
It can be read and printed by

\begin{Schunk}
\begin{Sinput}
> library(ctv)
> x <- read.ctv(file.path(.find.package("ctv"), "ctv", "Econometrics.ctv"))
\end{Sinput}
\begin{Soutput}
Loading required package: XML
\end{Soutput}
\begin{Sinput}
> x
\end{Sinput}
\begin{Soutput}
CRAN Task View
--------------
Name:       Econometrics
Topic:      Computational Econometrics
Maintainer: Achim Zeileis

Packages:   bayesm, betareg, bivpois, car*, Design, dse, dyn, dynlm, Ecdat,
            fCalendar, Hmisc, ineq, its, lmtest*, Matrix, micEcon, MNP, nlme,
            pscl, pwt, quantreg, reldist, sandwich*, segmented, sem, SparseM,
            strucchange, systemfit, tseries*, urca*, uroot, VR, zicounts, zoo*
            (* = core package)
\end{Soutput}
\end{Schunk}

The resulting object could be transformed into HTML by \verb/ctv2html(x)/.
Once this view has been installed at a CRAN-style repository, it can be installed by
the user by \verb/install.views("Econometrics")/.

\end{document}
